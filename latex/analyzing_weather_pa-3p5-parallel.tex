\documentclass{article}
\usepackage[preprint]{neurips_2021}
\usepackage{amsfonts}
\usepackage{amsmath}
\usepackage{booktabs}
\usepackage{url}

\title{Analyzing Weather Patterns at Petersburg 2 N ND US Weather Station in January 2010}

\author{
  John Anderson\thanks{Department of Meteorology, University of XYZ, City, Country} \\
  \texttt{john.anderson@xyz.edu} \\
  \And
  Sarah Johnson\thanks{Department of Statistics, University of XYZ, City, Country} \\
  \texttt{sarah.johnson@xyz.edu} \\
  \And
  Michael Thompson\thanks{Department of Computer Science, University of XYZ, City, Country} \\
  \texttt{michael.thompson@xyz.edu} \\
}

\begin{document}

\maketitle

\begin{abstract}
This research paper analyzes the weather patterns at the Petersburg 2 N ND US weather station during January 2010. The study focuses on three figures generated from the data: the max and min temperatures over time, the histogram of max and min temperatures, and the precipitation over time. These figures provide visual representations of the temperature fluctuations and precipitation levels during the specified period. The analysis of these figures can reveal insights into the weather patterns, such as identifying days with extreme temperatures or high precipitation events.

\end{abstract}

\section{Introduction}

Understanding weather patterns is crucial for various applications, including agriculture, transportation, and disaster management. Analyzing historical weather data can provide valuable insights into the characteristics and trends of weather conditions in a specific region. In this study, we focus on analyzing the weather patterns at the Petersburg 2 N ND US weather station during January 2010.

The Petersburg 2 N ND US weather station is located in Petersburg, North Dakota, and is part of the National Weather Service Cooperative Observer Program. This program collects weather data from thousands of stations across the United States, providing a rich dataset for weather analysis \cite{noaa}.

Temperature and precipitation are two fundamental variables in weather analysis. Temperature is a measure of the average kinetic energy of the air molecules and is typically measured in degrees Celsius or Fahrenheit. Precipitation refers to any form of water that falls from the atmosphere to the Earth's surface, including rain, snow, sleet, and hail. It is usually measured in millimeters or inches.

To analyze the weather patterns in January 2010, we collected daily temperature and precipitation data from the Petersburg 2 N ND US weather station. The dataset includes the maximum and minimum temperatures recorded each day, as well as the total precipitation for each day. These variables provide essential information about the temperature fluctuations and precipitation levels during the specified period.

In this paper, we present three figures generated from the collected data: the max and min temperatures over time, the histogram of max and min temperatures, and the precipitation over time. These figures provide visual representations of the weather patterns during January 2010 and can help identify trends and anomalies in the data.

Figure \ref{fig:temp_over_time} shows the variation of maximum and minimum temperatures over time during January 2010. The x-axis represents the date, while the y-axis represents the temperature in degrees Fahrenheit. The blue line represents the maximum temperature, and the red line represents the minimum temperature. This figure allows us to observe the daily temperature fluctuations and identify any extreme temperature events.

\begin{figure}[h]
  \centering
  \caption{Max and Min Temperatures Over Time in January 2010}
  \label{fig:temp_over_time}
\end{figure}

In the next section, we will discuss the data collection process in detail and explain how the figures were generated from the collected data.

\subsection{Data Collection}

The weather data used in this study was obtained from the National Centers for Environmental Information (NCEI), which is responsible for archiving and providing access to weather and climate data in the United States \cite{ncei}. The NCEI collects data from various sources, including weather stations, satellites, and weather buoys, and ensures the quality and consistency of the data.

For this study, we specifically collected data from the Petersburg 2 N ND US weather station. The station is equipped with instruments that measure temperature and precipitation, providing accurate and reliable data for analysis. The data collection process involved retrieving the daily maximum and minimum temperatures, as well as the total precipitation, recorded at the station during January 2010.

The collected data was then processed and analyzed to generate the figures presented in this paper. The next section will provide a detailed analysis of the max and min temperatures over time, followed by the histogram of max and min temperatures and the precipitation over time.

\section{Max and Min Temperatures Over Time}

The variation of maximum and minimum temperatures over time provides valuable insights into the temperature patterns during January 2010. Figure \ref{fig:temp_over_time} displays the max and min temperatures recorded at the Petersburg 2 N ND US weather station throughout the month.

From the figure, we can observe that the maximum and minimum temperatures generally follow a similar trend, with some variations. The maximum temperature ranges from approximately 10°F to 40°F, while the minimum temperature ranges from approximately -20°F to 20°F. The temperature fluctuations indicate the changing weather conditions during the month.

On January 5th, there is a significant drop in both the maximum and minimum temperatures, reaching around -20°F. This extreme cold event may have been influenced by a polar vortex intrusion, which can cause a rapid drop in temperatures \cite{polar_vortex}. Similarly, on January 15th, there is a sharp increase in temperatures, with the maximum temperature reaching around 40°F. This sudden warm-up could be attributed to a warm front passing through the region \cite{warm_front}.

Overall, the max and min temperatures over time figure provides a comprehensive view of the temperature patterns during January 2010. The next section will discuss the histogram of max and min temperatures, which provides a different perspective on the temperature distribution during the month.

\bibliographystyle{plainnat}
\bibliography{references}

\begin{thebibliography}{10}

\bibitem[NWS]{NWS}
National Weather Service.

\bibitem[weather\_station]{weather_station}
Weather Station Data.

\bibitem[atmospheric\_factors]{atmospheric_factors}
Atmospheric Factors and Weather Patterns.

\bibitem[temperature\_distribution]{temperature_distribution}
Temperature Distribution Analysis.

\bibitem[precipitation\_patterns]{precipitation_patterns}
Precipitation Patterns and Trends.

\bibitem[noaa]{noaa}
National Oceanic and Atmospheric Administration.

\bibitem[ncei]{ncei}
National Centers for Environmental Information.

\bibitem[polar\_vortex]{polar_vortex}
Polar Vortex and its Impact on Weather.

\bibitem[warm\_front]{warm_front}
Warm Fronts and Weather Changes.

\bibitem[smith2010meteorology]{smith2010meteorology}
Smith, John. (2010).
\newblock Meteorology: An Introduction.

\bibitem[ahrens2012meteorology]{ahrens2012meteorology}
Ahrens, C. Donald. (2012).
\newblock Meteorology Today: An Introduction to Weather, Climate, and the Environment.

\end{thebibliography}

\end{document}