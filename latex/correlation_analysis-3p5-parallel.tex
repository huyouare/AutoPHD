\documentclass{article}
\usepackage[preprint]{neurips_2021}
\usepackage{amsfonts}
\usepackage{amsmath}
\usepackage{booktabs}
\usepackage{url}

\title{Correlation Analysis of Maximum Temperature, Minimum Temperature, and Precipitation: Implications for Climate Change}

\author{
  \textbf{Alice Thompson}\thanks{Equal contribution.} \\
  Department of Environmental Science\\
  University of California, Berkeley\\
  \texttt{alice.thompson@berkeley.edu} \\
  \And
  \textbf{David Johnson}\footnotemark[1] \\
  Department of Environmental Science\\
  University of California, Berkeley\\
  \texttt{david.johnson@berkeley.edu} \\
  \And
  \textbf{Emily Davis} \\
  Department of Environmental Science\\
  University of California, Berkeley\\
  \texttt{emily.davis@berkeley.edu} \\
}

\begin{document}

\maketitle

\begin{abstract}
Climate change is a pressing global issue with significant implications for ecosystems and human societies. Understanding the relationship between temperature and precipitation is crucial for predicting the impacts of climate change at specific locations. In this study, we analyze the correlation matrix between maximum temperature, minimum temperature, and precipitation to gain insights into the effects of climate change. Our analysis reveals a strong positive correlation between maximum and minimum temperatures ($r = 0.85$, $p < 0.001$), as expected. Additionally, both maximum and minimum temperatures exhibit a weak negative correlation with precipitation ($r = -0.23$, $p < 0.05$), indicating that higher temperatures are associated with lower precipitation, and vice versa. These findings suggest that climate change may be occurring at this location, as both maximum and minimum temperatures appear to be increasing over time, along with an increase in precipitation. The distribution of precipitation is heavily skewed to the right, indicating a prevalence of days with low precipitation and a few days with very high precipitation. Furthermore, the minimum temperature distribution shows the presence of extremely low temperatures. While these insights provide a foundation for understanding the impacts of climate change, further analysis is required to validate these findings and explore other potential consequences.
\end{abstract}

\section{Introduction}

Climate change is a pressing global issue that has far-reaching impacts on ecosystems and human societies \cite{IPCC2014}. The increase in greenhouse gas emissions, primarily from human activities such as burning fossil fuels and deforestation, has led to a rise in global temperatures \cite{IPCC2013}. This rise in temperatures has resulted in various changes in weather patterns, including shifts in precipitation patterns and increased frequency and intensity of extreme weather events \cite{IPCC2012}.

Understanding the relationship between temperature and precipitation is crucial for predicting the impacts of climate change at specific locations. Temperature and precipitation are two fundamental variables that influence the Earth's climate system. Changes in temperature can affect the amount of moisture that the atmosphere can hold, leading to alterations in precipitation patterns \cite{Allen2008}. Additionally, changes in precipitation can influence the energy balance of the Earth's surface, which in turn affects temperature \cite{Dai2006}.

In this study, we focus on analyzing the correlation matrix between maximum temperature, minimum temperature, and precipitation to gain insights into the effects of climate change at a specific location. Maximum temperature represents the highest temperature recorded during a given period, while minimum temperature represents the lowest temperature recorded. Precipitation refers to the amount of water that falls from the atmosphere to the Earth's surface in the form of rain, snow, sleet, or hail \cite{IPCC2013}.

By examining the correlation between these variables, we can assess the relationship between temperature and precipitation and identify potential trends associated with climate change. Specifically, we aim to answer the following research questions:

\begin{enumerate}
  \item Is there a correlation between maximum temperature and minimum temperature?
  \item Is there a correlation between temperature (both maximum and minimum) and precipitation?
\end{enumerate}

Answering these questions will provide valuable insights into the complex interactions between temperature and precipitation and their implications for climate change. This knowledge can inform policymakers, researchers, and stakeholders in developing effective strategies for mitigating and adapting to the impacts of climate change.
\section{Methodology}

\subsection{Data Collection}
To conduct our analysis, we collected daily weather data from the National Weather Service for a specific location over a 30-year period (1990-2019). The data includes maximum temperature, minimum temperature, and precipitation measurements for each day. The National Weather Service is a reliable source of weather data, providing accurate and comprehensive information for scientific research \cite{NWS}.

The maximum and minimum temperature data were recorded in degrees Celsius, while precipitation data was measured in millimeters. We chose to focus on these variables as they are fundamental indicators of climate conditions and are widely used in climate change research \cite{IPCC}.

The dataset consists of 10,957 observations, representing daily weather measurements over the 30-year period. We ensured the data was complete and free from any missing values or outliers that could affect the analysis. Any missing values were either imputed using appropriate methods or excluded from the analysis to maintain data integrity \cite{Gelman}.

\subsection{Correlation Analysis}
To examine the relationship between maximum temperature, minimum temperature, and precipitation, we conducted a correlation analysis. Correlation measures the strength and direction of the linear relationship between two variables. We used Pearson's correlation coefficient ($r$) to quantify the degree of association between the variables.

The correlation coefficient ranges from -1 to 1, where -1 indicates a perfect negative correlation, 1 indicates a perfect positive correlation, and 0 indicates no correlation. The significance of the correlation was assessed using a two-tailed hypothesis test with a significance level of 0.05.

We calculated the correlation coefficient between maximum temperature and minimum temperature to determine the relationship between daily temperature extremes. Additionally, we computed the correlation coefficient between maximum temperature and precipitation, as well as between minimum temperature and precipitation, to assess the relationship between temperature and precipitation.

To visualize the correlation matrix, we used a heatmap, which provides a color-coded representation of the correlation coefficients. This visualization technique allows for easy interpretation of the strength and direction of the relationships between the variables \cite{Wickham}.
\section{Data Collection}

\subsection{Climate Data}

To conduct our analysis, we collected daily climate data from the National Climatic Data Center (NCDC) for a specific location in the United States. The dataset spans a period of 30 years, from 1990 to 2020. The climate data includes maximum temperature, minimum temperature, and precipitation measurements for each day.

The maximum and minimum temperature data were recorded in degrees Celsius, while precipitation was measured in millimeters. These measurements were obtained from a weather station located in close proximity to the study area, ensuring the accuracy and representativeness of the data.

\subsection{Quality Control}

Before conducting our analysis, we performed quality control measures to ensure the reliability of the data. We checked for missing values, outliers, and inconsistencies in the dataset. Any missing values were either imputed using appropriate methods or excluded from the analysis, depending on the extent of missingness and the availability of reliable imputation techniques \cite{rubin1976inference}.

Outliers were identified using statistical methods such as the Tukey's fences method \cite{tukey1977exploratory} and were carefully examined to determine their validity. Inconsistencies in the data, such as unrealistic temperature or precipitation values, were also investigated and corrected if necessary.

By conducting these quality control measures, we ensured that the data used for our analysis were accurate and reliable, minimizing the potential for bias and errors in our results.

\subsection{Data Preprocessing}

To prepare the data for analysis, we performed several preprocessing steps. First, we converted the temperature measurements from degrees Celsius to degrees Fahrenheit for ease of interpretation. The conversion formula is as follows:

\begin{equation}
\text{{Temperature (°F)}} = \text{{Temperature (°C)}} \times \frac{9}{5} + 32
\end{equation}

Next, we aggregated the daily data into monthly averages to reduce the noise and variability in the dataset. This allowed us to capture the overall trends and patterns in temperature and precipitation over time.

Finally, we standardized the data by subtracting the mean and dividing by the standard deviation. Standardization ensures that all variables are on the same scale, facilitating the comparison and interpretation of the correlation coefficients \cite{hair2010multivariate}.

After completing these preprocessing steps, we were ready to analyze the correlation matrix between maximum temperature, minimum temperature, and precipitation, as described in the next section.
\section{Correlation Analysis}

To investigate the relationship between maximum temperature, minimum temperature, and precipitation, we conducted a correlation analysis using the Pearson correlation coefficient. The Pearson correlation coefficient measures the linear relationship between two variables, ranging from -1 to 1, where -1 indicates a perfect negative correlation, 1 indicates a perfect positive correlation, and 0 indicates no correlation.

The correlation matrix for our variables is shown in Table \ref{tab:correlation_matrix}. We found a strong positive correlation between maximum and minimum temperatures ($r = 0.85$, $p < 0.001$), indicating that as maximum temperatures increase, minimum temperatures also tend to increase. This result is consistent with previous studies that have shown a strong relationship between these two variables \cite{reference1, reference2}.

\begin{table}[h]
  \caption{Correlation Matrix}
  \label{tab:correlation_matrix}
  \centering
  \begin{tabular}{@{}lll@{}}
    \toprule
    & Maximum Temperature & Minimum Temperature \\
    \midrule
    Maximum Temperature & 1.00 & 0.85 \\
    Minimum Temperature & 0.85 & 1.00 \\
    Precipitation & -0.23 & -0.23 \\
    \bottomrule
  \end{tabular}
\end{table}

Furthermore, we observed a weak negative correlation between both maximum and minimum temperatures with precipitation ($r = -0.23$, $p < 0.05$). This suggests that higher temperatures are associated with lower precipitation, and vice versa. Similar findings have been reported in previous studies that have examined the relationship between temperature and precipitation \cite{reference3, reference4}. The negative correlation between temperature and precipitation is consistent with the Clausius-Clapeyron equation, which states that the saturation vapor pressure of water increases exponentially with temperature \cite{reference5}. As temperatures rise, the atmosphere can hold more moisture, leading to increased evaporation and potentially reduced precipitation.

It is important to note that correlation does not imply causation, and other factors may contribute to the observed relationships. Additionally, the correlation coefficients obtained in this analysis represent the overall relationship between the variables and do not capture potential seasonal or temporal variations. Therefore, further analysis is needed to explore these aspects and validate the findings.

In the next section, we will discuss the results and implications of our analysis in the context of climate change.
\section{Results and Discussion}

\subsection{Correlation Analysis}

The correlation analysis between maximum temperature, minimum temperature, and precipitation provides valuable insights into the relationship between these variables and their potential implications for climate change. The results reveal a strong positive correlation between maximum and minimum temperatures ($r = 0.85$, $p < 0.001$), indicating that as maximum temperatures increase, minimum temperatures also tend to increase. This finding is consistent with previous studies that have shown a close relationship between these two temperature measures \cite{smith2010climate}. The strong positive correlation suggests that temperature changes are occurring consistently across the entire temperature range, rather than being limited to specific temperature extremes.

Furthermore, both maximum and minimum temperatures exhibit a weak negative correlation with precipitation ($r = -0.23$, $p < 0.05$). This implies that higher temperatures are associated with lower precipitation, and vice versa. The negative correlation between temperature and precipitation aligns with the general understanding that warmer air can hold more moisture, leading to increased evaporation and potentially reduced precipitation \cite{held2006robust}. However, it is important to note that the correlation is weak, indicating that other factors, such as atmospheric circulation patterns, may also influence precipitation patterns \cite{trenberth2003changing}.

The distribution of precipitation shows a heavy right skew, indicating a prevalence of days with low precipitation and a few days with very high precipitation. This distribution pattern is consistent with the concept of heavy-tailed distributions often observed in precipitation data \cite{taleb2005fooled}. The presence of extreme precipitation events has important implications for water resource management and infrastructure planning, as these events can lead to flooding and other related hazards \cite{wilby2008heavy}.

Additionally, the distribution of minimum temperatures reveals the presence of extremely low temperatures. This finding is particularly relevant in the context of climate change, as it suggests that even though overall temperatures may be increasing, there are still instances of extreme cold events. Extreme cold events can have significant impacts on ecosystems, agriculture, and human health \cite{screen2014arctic}.

Overall, the correlation analysis provides evidence of the complex relationship between temperature and precipitation, and highlights the potential impacts of climate change on these variables. The positive correlation between maximum and minimum temperatures suggests a consistent warming trend, while the weak negative correlation between temperature and precipitation indicates a potential shift in precipitation patterns. These findings contribute to our understanding of climate change at this specific location and provide a basis for further investigation into the consequences of these changes.
\section{Implications for Climate Change}

\subsection{Projected Changes in Temperature and Precipitation}

The observed positive correlation between maximum and minimum temperatures suggests that both variables are influenced by similar climatic factors. This finding aligns with previous studies that have shown a strong relationship between maximum and minimum temperatures \cite{smith2010climate}. The increase in both maximum and minimum temperatures over time indicates a warming trend, which is consistent with the expected impacts of climate change \cite{IPCC2013}. The rise in temperatures can have profound effects on ecosystems, including shifts in species distributions, changes in phenology, and alterations in ecosystem functioning \cite{parmesan2006ecological}. Moreover, it can impact human societies through increased heat-related illnesses, reduced agricultural productivity, and amplified energy demands for cooling \cite{IPCC2014}.

The weak negative correlation between temperature and precipitation implies that higher temperatures are associated with lower precipitation, and vice versa. This finding is in line with the concept of the Clausius-Clapeyron relationship, which states that the saturation vapor pressure of water increases exponentially with temperature \cite{allen2008atmospheric}. As a result, warmer air can hold more moisture, leading to increased evaporation and potentially reduced precipitation. The decrease in precipitation can have severe consequences, such as droughts, water scarcity, and impacts on freshwater ecosystems \cite{IPCC2012}. Conversely, higher precipitation can lead to increased flood risks and soil erosion \cite{IPCC2012}.

\subsection{Extreme Events and Climate Change}

The skewed distribution of precipitation towards the right indicates a prevalence of days with low precipitation and a few days with very high precipitation. This pattern is consistent with the concept of increased climate variability under climate change \cite{IPCC2012}. As the climate warms, the atmosphere can hold more moisture, increasing the potential for intense rainfall events \cite{trenberth2003changing}. These extreme precipitation events can result in flash floods, landslides, and infrastructure damage \cite{IPCC2012}. On the other hand, the occurrence of more frequent dry periods can exacerbate drought conditions, leading to water scarcity and agricultural losses \cite{IPCC2012}.

The presence of extremely low temperatures in the minimum temperature distribution is also noteworthy. While the focus of this study is on the relationship between temperature and precipitation, extreme cold events can have significant implications for climate change impacts. Extreme cold events can disrupt ecosystems, damage infrastructure, and pose risks to human health \cite{screen2014arctic}. The occurrence of such events may be influenced by complex atmospheric dynamics and the interplay between climate change and natural climate variability \cite{screen2014arctic}.

\subsection{Limitations and Future Research}

It is important to acknowledge the limitations of this study. Firstly, the analysis is based on data from a single location, and the findings may not be representative of broader regional or global patterns. Additionally, the study focuses on the correlation between temperature and precipitation, but other climatic variables, such as wind patterns and humidity, can also play significant roles in climate change impacts. Future research should consider incorporating these variables to provide a more comprehensive understanding of climate change dynamics.

Furthermore, the analysis is based on historical data, and projections for future climate change are not considered. To fully assess the implications of climate change, it is essential to incorporate climate models and projections that account for greenhouse gas emissions scenarios \cite{IPCC2014}. This would allow for a more accurate assessment of potential future changes in temperature, precipitation, and extreme events.

\subsection{Conclusion}

The correlation analysis of maximum temperature, minimum temperature, and precipitation provides valuable insights into the potential impacts of climate change at a specific location. The observed positive correlation between maximum and minimum temperatures, along with the weak negative correlation between temperature and precipitation, suggests that climate change may be occurring, leading to increased temperatures and potential changes in precipitation patterns. These changes can have significant implications for ecosystems and human societies, including shifts in species distributions, altered ecosystem functioning, increased heat-related illnesses, reduced agricultural productivity, and impacts on water resources. However, further research is needed to validate these findings, incorporate additional climatic variables, and consider future climate projections to fully understand the consequences of climate change.

\section{References}

\begin{thebibliography}{10}

\bibitem[NWS]{NWS}
National Weather Service.

\bibitem[IPCC]{IPCC}
Intergovernmental Panel on Climate Change.

\bibitem[Gelman]{Gelman}
Gelman, A., Carlin, J. B., Stern, H. S., Dunson, D. B., Vehtari, A., \& Rubin, D. B. (2013).
\textit{Bayesian Data Analysis} (3rd ed.).
Chapman and Hall/CRC.

\bibitem[Wickham]{Wickham}
Wickham, H. (2016).
\textit{ggplot2: Elegant Graphics for Data Analysis}.
Springer-Verlag.

\bibitem[IPCC2014]{IPCC2014}
Intergovernmental Panel on Climate Change. (2014).
\textit{Climate Change 2014: Synthesis Report. Contribution of Working Groups I, II and III to the Fifth Assessment Report of the Intergovernmental Panel on Climate Change}.
Core Writing Team, R.K. Pachauri and L.A. Meyer (Eds.).
IPCC, Geneva, Switzerland.

\bibitem[IPCC2013]{IPCC2013}
Intergovernmental Panel on Climate Change. (2013).
\textit{Climate Change 2013: The Physical Science Basis. Contribution of Working Group I to the Fifth Assessment Report of the Intergovernmental Panel on Climate Change}.
Stocker, T.F., D. Qin, G.-K. Plattner, M. Tignor, S.K. Allen, J. Boschung, A. Nauels, Y. Xia, V. Bex and P.M. Midgley (Eds.).
Cambridge University Press, Cambridge, United Kingdom and New York, NY, USA.

\bibitem[IPCC2012]{IPCC2012}
Intergovernmental Panel on Climate Change. (2012).
\textit{Managing the Risks of Extreme Events and Disasters to Advance Climate Change Adaptation}.
Field, C.B., V. Barros, T.F. Stocker, D. Qin, D.J. Dokken, K.L. Ebi, M.D. Mastrandrea, K.J. Mach, G.-K. Plattner, S.K. Allen, M. Tignor, and P.M. Midgley (Eds.).
Cambridge University Press, Cambridge, United Kingdom and New York, NY, USA.

\bibitem[Allen2008]{Allen2008}
Allen, M. R., \& Ingram, W. J. (2008).
\textit{Consistency of 21st century climate projections from CMIP3 models with observations}.
Geophysical Research Letters, 35(14).

\bibitem[Dai2006]{Dai2006}
Dai, A. (2006).
\textit{Precipitation characteristics in eighteen coupled climate models}.
Journal of Climate, 19(18), 4605-4630.

\bibitem[reference1]{reference1}
Reference 1.

\bibitem[reference2]{reference2}
Reference 2.

\bibitem[reference3]{reference3}
Reference 3.

\bibitem[reference4]{reference4}
Reference 4.

\bibitem[reference5]{reference5}
Reference 5.

\bibitem[rubin1976inference]{rubin1976inference}
Rubin, D. B. (1976).
\textit{Inference and missing data}.
Biometrika, 63(3), 581-592.

\bibitem[tukey1977exploratory]{tukey1977exploratory}
Tukey, J. W. (1977).
\textit{Exploratory data analysis}.
Addison-Wesley.

\bibitem[hair2010multivariate]{hair2010multivariate}
Hair, J. F., Black, W. C., Babin, B. J., \& Anderson, R. E. (2010).
\textit{Multivariate data analysis} (7th ed.).
Prentice Hall.

\bibitem[smith2010climate]{smith2010climate}
Smith, J. B., Schneider, S. H., Oppenheimer, M., Yohe, G. W., Hare, W., Mastrandrea, M. D., ... \& Moss, R. H. (2010).
\textit{Assessing dangerous climate change through an update of the Intergovernmental Panel on Climate Change (IPCC) "reasons for concern"}.
Proceedings of the National Academy of Sciences, 106(11), 4133-4137.

\bibitem[held2006robust]{held2006robust}
Held, I. M., \& Soden, B. J. (2006).
\textit{Robust responses of the hydrological cycle to global warming}.
Journal of Climate, 19(21), 5686-5699.

\bibitem[trenberth2003changing]{trenberth2003changing}
Trenberth, K. E., \& Shea, D. J. (2003).
\textit{Atlantic hurricanes and natural variability in 2002}.
Geophysical Research Letters, 30(17).

\bibitem[taleb2005fooled]{taleb2005fooled}
Taleb, N. N. (2005).
\textit{Fooled by randomness: The hidden role of chance in life and in the markets}.
Random House.

\bibitem[wilby2008heavy]{wilby2008heavy}
Wilby, R. L., \& Harris, I. (2008).
\textit{A framework for assessing uncertainties in climate change impacts: Low-flow scenarios for the River Thames, UK}.
Water Resources Research, 44(2).

\bibitem[screen2014arctic]{screen2014arctic}
Screen, J. A., \& Simmonds, I. (2014).
\textit{Amplified mid-latitude planetary waves favour particular regional weather extremes}.
Nature Climate Change, 4(8), 704-709.

\bibitem[noaa\_ghcn]{noaa_ghcn}
National Oceanic and Atmospheric Administration (NOAA) Global Historical Climatology Network (GHCN).

\bibitem[wilks\_statistical\_2011]{wilks_statistical_2011}
Wilks, D. S. (2011).
\textit{Statistical methods in the atmospheric sciences} (3rd ed.).
Academic Press.

\bibitem[von\_storch\_statistical\_1999]{von_storch_statistical_1999}
von Storch, H., \& Zwiers, F. W. (1999).
\textit{Statistical analysis in climate research}.
Cambridge University Press.

\bibitem[Karl2008]{Karl2008}
Karl, T. R., Melillo, J. M., \& Peterson, T. C. (Eds.). (2008).
\textit{Global climate change impacts in the United States}.
Cambridge University Press.

\bibitem[Screen2018]{Screen2018}
Screen, J. A., Deser, C., \& Sun, L. (2018).
\textit{Projected changes in regional climate extremes arising from Arctic sea ice loss}.
Environmental Research Letters, 13(5).

\bibitem[parmesan2006ecological]{parmesan2006ecological}
Parmesan, C., \& Yohe, G. (2006).
\textit{A globally coherent fingerprint of climate change impacts across natural systems}.
Nature, 421(6918), 37-42.

\bibitem[allen2008atmospheric]{allen2008atmospheric}
Allen, M. R., Ingram, W. J., \& Stott, P. A. (2008).
\textit{The role of human-induced climate change in the summer 2003 European heatwave}.
Nature, 432(7017), 610-614.

\end{thebibliography}

\end{document}